
% This LaTeX was auto-generated from an M-file by MATLAB.
% To make changes, update the M-file and republish this document.
\documentclass[12pt,preprint,ntfdMod]{elsarticle}
\usepackage{graphicx}
\usepackage{color}
\usepackage{mcode}
\usepackage{amsmath}
\usepackage{amsfonts}
\usepackage{hyperref}
\usepackage[protrusion=true,expansion]{microtype}
\usepackage{esint}
\hypersetup{
	bookmarks=true,         % show bookmarks bar?
	unicode=false,          % non-Latin characters in Acrobat’s bookmarks
	pdftoolbar=true,        % show Acrobat’s toolbar?
	pdfmenubar=true,        % show Acrobat’s menu?
	pdffitwindow=false,     % window fit to page when opened
	pdfstartview={FitH},    % fits the width of the page to the window
	pdftitle={My title},    % title
	pdfauthor={Author},     % author
	pdfsubject={Subject},   % subject of the document
	pdfcreator={Creator},   % creator of the document
	pdfproducer={Producer}, % producer of the document
	pdfkeywords={keyword1} {key2} {key3}, % list of keywords
	pdfnewwindow=true,      % links in new window
	colorlinks=true,       % false: boxed links; true: colored links
	linkcolor=red,          % color of internal links
	citecolor=green,        % color of links to bibliography
	filecolor=magenta,      % color of file links
	urlcolor=cyan           % color of external links
}

\sloppy
\definecolor{lightgray}{gray}{0.5}
\setlength{\parindent}{0pt}
\author{Felix Dietzsch}
\begin{document}
\maketitle

    
    
\section*{DOCUMENT TITLE}

\begin{par}
INTRODUCTORY TEXT
\end{par} \vspace{1em}

\subsection*{Contents}

\begin{itemize}
\setlength{\itemsep}{-1ex}
   \item Clear complete workspace
   \item Read data files
   \item 3D
   \item Set some neccessary parameters
   \item 3D
   \item Compute FFT
   \item Compute correlations
   \item Compute length scales
   \item Spectrum computation
   \item Compute 1D spectrum
   \item Compute 3D spectrum
   \item Compute $\kappa$ vector
   \item compute 1D spectrum
\end{itemize}


\subsection*{Clear complete workspace}

\begin{par}
Its always a good idea to clear the complete workspace and the command window also closing all figures might be helpful. You may also use the header defin some neccessary flags distinguishing bewteen different data sets.
\end{par} \vspace{1em}
\begin{verbatim}
close all
clear all
clc

flag='2D';
datadir='data';
\end{verbatim}


\subsection*{Read data files}

\begin{par}
Read in the data files and measure the time for reading. The output of the tic/toc block is in seconds. What you should get from the tic/toc block is that most of the time is spend during data I/O. The actual computation needs only ??? of the time of the I/O operations.
\end{par} \vspace{1em}


\subsection*{3D}

\begin{verbatim}
if (strcmp('3D',flag))
    tic; % enable timer
    uvel=importdata([datadir,'/',flag,'/uvel']);
    vvel=importdata([datadir,'/',flag,'/vvel']);
    wvel=importdata([datadir,'/',flag,'/wvel']);
    time_reading = toc; % end timer
end
%%% 2D
if (strcmp('2D',flag))
    tic;
    uvel=importdata([datadir,'/',flag,'/uvel']);
    vvel=importdata([datadir,'/',flag,'/vvel']);
    time_reading = toc;
end
\end{verbatim}


\subsection*{Set some neccessary parameters}

\begin{par}
For further computations it is important to define some parmeters of the DNS simulation such as domain size, grid spacing, and the number of grid points.
\end{par} \vspace{1em}


\subsection*{3D}

\begin{verbatim}
if (strcmp('3D',flag))
    dim=256; % number of points in one dimension
    Lx=5e-3; % domain size
    Ly=Lx;
    Lz=Lx;
    dx=Lx/dim; % grid spacing
    dy=dx;
    dz=dx;
    nu=1.7e-5; % viscosity
    u=reshape(uvel,dim,dim,dim); % reshape arrays to have them in 3D
    v=reshape(vvel,dim,dim,dim);
    w=reshape(wvel,dim,dim,dim);
end
%%% 2D
if (strcmp('2D',flag))
    dim=1024; % number of points in one dimension
    Lx=1E-2;  % domain size
    Ly=Lx;
    dx=Lx/dim; % grid spacing
    dy=dx;
    u=reshape(uvel,dim,dim); % reshape arrays to have them in 2D
    v=reshape(vvel,dim,dim);
end
\end{verbatim}


\subsection*{Compute FFT}

\begin{par}
This is the most important part of the script. Since the performance of an actual DFT is rather bad the preferred choice is a FFT. The FFT approach is fastest if the data set to be transformed has a size that is a multiple of two. Thats why the function \textbf{nextpow2} is used to get the next powert of two approximating the dimension \textit{dim} of the data set. As a consequence the data set is zero padded or truncated. \textit{Since the output of an FFT operation is symmetric we only need to save half the transform}.
\end{par} \vspace{1em}
\begin{par}

  \begin{equation}
     \Phi_{ij}(\kappa)=\frac{1}{(2\,\pi)^3}\iiint\limits^{\infty}_{-\infty}
                          R_{ij}(\mathbf{r})\,\mathrm{e}^{-i\mathrm{\kappa}r}
                          \,\mathrm{d}\mathbf{r}
  \end{equation}
After the transformation of all velocity components we have to compute
the velocity correlation tensor $\Phi$ . From theory we know
  \begin{equation}
  (u_i*u_j)=\int\limits_{-\infty}^{\infty}u_i^{*}(\mathbf{x})\,
                  u_j(\mathbf{x}+\mathbf{r})\,\mathrm{d}\mathbf{r}.
  \end{equation}
Since all our data sets are transformed (and we are in the Fourier space)
the last expression can be simply computed by multiplying
  \begin{equation}
      \mathfrak{F}\left\{u_i*u_j\right\} = \alpha\cdot
          \left\{\mathfrak{F}\left\{u_i\right\}\right\}^{*}\cdot
                 \mathfrak{F}\left\{u_j\right\},
  \end{equation}
where $\alpha$ is a normalization factor.

\end{par} \vspace{1em}
\begin{verbatim}
if (strcmp('3D',flag))
    tic; % start timer
    NFFT = 2.^nextpow2(size(u)); % next power of 2 fitting the length of u
    u_fft=fftn(u,NFFT)./(2*pi)^3; %2 pi --> definition of FFT
    %
    NFFT = 2.^nextpow2(size(v));
    v_fft=fftn(v,NFFT)./(2*pi)^3;
    %
    NFFT = 2.^nextpow2(size(w));
    w_fft=fftn(w,NFFT)./(2*pi)^3;
    time_fft=toc; % get final time for all transformations

    phi_x=u_fft.*conj(u_fft)/dim^6; % compute velocity correlation tensor
    phi_y=v_fft.*conj(v_fft)/dim^6;
    phi_z=w_fft.*conj(w_fft)/dim^6;
end
if (strcmp('2D',flag))
    tic; %start timer
    NFFT = 2.^nextpow2(size(u));
    u_fft=fft2(u,NFFT(1),NFFT(2))./(2*pi)^2; %2 pi --> definition of FFT
    %
    NFFT = 2.^nextpow2(size(v));
    v_fft=fft2(v,NFFT(1),NFFT(2))./(2*pi)^2;
    %
    phi_x=u_fft.*conj(u_fft)/size(u,1).^2/size(u,2).^2;
    phi_y=v_fft.*conj(v_fft)/size(v,1).^2/size(v,2).^2;
end
\end{verbatim}


\subsection*{Compute correlations}

\begin{par}
Computing a correlation can be a tedious work (requireing tremendeous effort) especially if you have large data sets. From theory it is well known that the multiplication of the transform of a data set and its complex conjugate are an accurate representation of the correlation function. Using the FFT approach this gives an enormeous speed advantage. Since we already computed the veloity correlation tensor we may use this result in order to compute the correlation tensor.
\end{par} \vspace{1em}
\begin{par}

  \begin{equation}
      R_{ij} = \frac{cov(U_i,U_j)}{\sqrt{\sigma_i^2\,\sigma_j^2}}
             = \frac{(u_i'-\mu_i)\,(u_j-\mu_j)}{\sqrt{\sigma_i^2\,\sigma_j^2}}
  \end{equation}

\end{par} \vspace{1em}
\begin{verbatim}
if (strcmp('3D',flag))
    R11=ifftn(u_fft.*conj(u_fft))/dim^3/std2(u)^2;
    R22=ifftn(u_fft.*conj(u_fft))/dim^3/std2(v)^2;
    R33=ifftn(u_fft.*conj(u_fft))/dim^3/std2(w)^2;
    %
    R11=R11(1:round(size(R11,1)/2),1,1);
    R22=R22(1:round(size(R22,1)/2),1,1);
    R33=R33(1:round(size(R33,1)/2),1,1);
    %
    r = linspace(0,Lx/2,dim/2)/(Lx/2);
end
if (strcmp('2D',flag))
\end{verbatim}
\begin{verbatim}
    NFFT = 2.^nextpow2(size(u_fft));
    R1 = ifft2(u_fft.*conj(u_fft),NFFT(1),NFFT(2))...
                ./NFFT(1)./NFFT(2)./std2(u)^2 ...
                .*(2*pi)^4; % scaling due to division by 2*pi
    %
    NFFT = 2.^nextpow2(size(v_fft));
    R2 = ifft2(v_fft.*conj(v_fft),NFFT(1),NFFT(2))...
                ./NFFT(1)./NFFT(2)./std2(v)^2 ...
                .*(2*pi)^4; % scaling due to division by 2*pi
    %
    R11 = (R1(1:round(size(R1,1)/2),1) + ...
           R2(1,1:round(size(R2,1)/2))')/2; % build the mean
    R22 = (R2(1:round(size(R2,1)/2),1) + ...
           R1(1,1:round(size(R1,1)/2))')/2;
    %
    r = linspace(0,Lx/2,dim/2)/(Lx/2); % get the radius
\end{verbatim}
\begin{par}

  From theory we know that the transverse correlation could also be
  computed from the longitudinal correlation by
  \begin{equation}
      g(r) = f + \frac{r}{2}\frac{\partial f}{\partial r}
  \end{equation}

\end{par} \vspace{1em}
\begin{verbatim}
    g_r = R11 + r'/2.*gradient(R11,max(diff(r)));
\end{verbatim}
\begin{verbatim}
end
plot(r,R11,r,R22,r,g_r)
legend('R11','R22','g_r');
h=line([0 1],[0 0],'Color',[0 0 0],'LineWidth',1.0);
% 2D graphs of correlation function
pcolor(fftshift(R1));shading interp;title('R11');
figure
pcolor(fftshift(R2));shading interp;title('R22');
\end{verbatim}

\includegraphics [width=4in]{spectrum_01.jpg}

\includegraphics [width=4in]{spectrum_02.jpg}


\subsection*{Compute length scales}

\begin{par}
Computing the length scales is rather easy. The longitudinal and transverse length scale are defined through
\end{par} \vspace{1em}
\begin{par}

  \begin{eqnarray}
      L_{11} &= \int\limits_0^{\infty}R_{11}\,\mathrm{d}r\\
      L_{22} &= \int\limits_0^{\infty}R_{22}\,\mathrm{d}r
  \end{eqnarray}
  Since our data is not represented in an analytical manner we may use a
  numerical integration routine. Matlab supporty only one numerical
  integration scheme, namely the Trapezoidal numerical integration. For
  more information about integration routines you can visit the
  \href{http://www.mathworks.de/support/solutions/en/data/1-1679J/index.html}
  {Mathworks Matlab}
  support page.

\end{par} \vspace{1em}
\begin{verbatim}
L11=trapz(r,R11);
L22=trapz(r,R22);
hold on
rectangle('Position',[0,0,L11,1],'LineWidth',2,'LineStyle','--')
\end{verbatim}

\includegraphics [width=4in]{spectrum_03.jpg}


\subsection*{Spectrum computation}

\begin{par}
In general the spectrum of a phyiscal quantity has three dimensions whereas the direction in wavenumber space is indicated by $\kappa_1$, $\kappa_2$ and $\kappa_3$. Opposed to this relatively extensive computation one also might get an idea of the spectral distribution calculating the one dimensional spectra. This is achieved by Fourier transforming the previously computed correlation functions.
\end{par} \vspace{1em}
\begin{par}

  \begin{equation}
      E_{ij}(\kappa_1) = \frac{1}{\pi} \int\limits_{-\infty}^{\infty}
                               \mathbf{R}_{ij}(e_1r_1)\,\mathrm{e}^{-i\kappa_1 r_1}
                                \mathrm{d}r_1
  \end{equation}

\end{par} \vspace{1em}


\subsection*{Compute 1D spectrum}

\begin{verbatim}
L=length(R11);
NFFT=2^nextpow2(L);
E11=fft(R11,NFFT)/L.*2/pi.*std2(u)^2;
%
L=length(R22);
NFFT=2^nextpow2(L);
E22=fft(R22,NFFT)/L.*2/pi;

f = linspace(0,1,NFFT)*2*pi/dx;

slope=1.5*664092^(2/3)*(f.^(-5/3));
% loglog(f,2*abs(spec_(1:NFFT/2+1)));
% hold on
% loglog(f,slope);
\end{verbatim}


\subsection*{Compute 3D spectrum}

\begin{par}
In order to avoid aliasing effects usually connected with a one dimensional spectrum it is also possible to produce correlations that involve all possible directions. The three dimensional Fourier transformation of such a correlation produces a spectrum that not only depends on a single wavenumber but on the wavenumber vector $\kappa_i$. Though the directional information contained in $\kappa_i$ eliminates the aliasing problem the complexity makes a physical reasoning impossible. For homogeneous isotropic turbulence the situation can be simplified by integrating the three dimensional spectrum over spherical shells.
\end{par} \vspace{1em}
\begin{par}

  \begin{equation}
      E(\kappa) = \oiint E(\boldsymbol\kappa)\mathrm{d}S(\kappa)
                = \oiint \frac{1}{2}\,Phi_{ii}(\boldsymbol\kappa)\mathrm{d}S(\kappa)
  \end{equation}
  Since the surface of a sphereis completly determined by its radius the
  surface integral can be solved analytically.
  \begin{equation}
      \oiint(\,)\mathrm{d}S(\kappa) = 4\pi\kappa^2\cdot(\,)
  \end{equation}
This leads to
  \begin{equation}
      E(|\kappa|) = \frac{1}{2}\,Phi_{ii}(|\boldsymbol\kappa|)
  \end{equation}

\end{par} \vspace{1em}
\begin{verbatim}
if (strcmp('3D',flag))
    phi = 0.5*(phi_x+phi_y+phi_z);
    phi = phi(1:round(size(phi_x,1)/2),...
              1:round(size(phi_y,1)/2),...
              1:round(size(phi_z,1)/2));
end
if (strcmp('2D',flag))
    phi = 0.5*phi_x+phi_y;
    phi = phi(1:round(size(phi_x,1)/2),...
              1:round(size(phi_y,1)/2));
end
\end{verbatim}


\subsection*{Compute $\kappa$ vector}

\begin{par}
From the previous section we know that the only independent we have in the ``system'' is the magnitude of the wave number vector, i.e. $\kappa=|\boldsymbol\kappa|=\sqrt{\kappa_1+\kappa_2+\kappa_3}$. Secondly we have to compute the sum $\Phi_{ii}=\Phi_{11}+\Phi_{22}+\Phi_{33}$ and take into account its dependence on $|\boldsymbol\kappa|$.
\end{par} \vspace{1em}
\begin{verbatim}
if (strcmp('3D',flag))
    maxdim = sqrt(dim^2*(2*pi/Lx)^2+dim^2*(2*pi/Ly)^2+dim^2*(2*pi/Lz)^2);
    E=zeros(uint64(sqrt(3*dim^2)),1);
    kk=zeros(uint64(sqrt(3*dim^2)),1);
    dim = size(phi,1);
    for k=1:dim
        for j=1:dim
            for i=1:dim
                kappa=sqrt(i*i*(2*pi/Lx)^2+j*j*(2*pi/Ly)^2+k*k*(2*pi/Lz)^2);
                kappa_pos=uint64(sqrt(i*i+j*j+k*k));
                E(kappa_pos) = E(kappa_pos) + phi(i,j,k);
                kk(kappa_pos) = kappa;
            end
        end
    end
    E=E*4*pi;
% E=E.*kk.^2;
end
if (strcmp('2D',flag))
    dim = size(phi,1);
    maxdim = sqrt(dim^2*(2*pi/Lx)^2+dim^2*(2*pi/Ly)^2);
    E=zeros(uint64(sqrt(dim^2+dim^2)),1);
    kk=zeros(uint64(sqrt(dim^2+dim^2)),1);
    bin_counter=zeros(uint64(sqrt(dim^2+dim^2)),1);
    for j=1:dim
        for i=1:dim
            kappa=sqrt(i*i*(2*pi/Lx)^2+j*j*(2*pi/Ly)^2);
            kappa_pos=uint64(sqrt(i*i+j*j));
            E(kappa_pos) = E(kappa_pos) + phi(i,j);
			bin_counter(kappa_pos) = bin_counter(kappa_pos) + 1;
            kk(kappa_pos) = kappa;
        end
    end
    EE=E*2*pi.*kk./bin_counter;
%     EEE = E*2*pi.*kk;
end
\end{verbatim}


\subsection*{compute 1D spectrum}

\begin{par}
close all
\end{par} \vspace{1em}
\begin{verbatim}
slope=1.5*664092^(2/3)*(kk.^(-5/3));
% test=importdata('INPUT/2D/CTRL_TURB_ENERGY');
%
% dissip=664092;
dissip=664092;
up=17;
L=Lx;
kkke=kk./(2*pi)*L;
kkkd=kk./(2*pi*100)*L;
VKP = 1.5*17^5/dissip.*(kkke).^4./(1+kkke.^2).^(17/6).*exp(-3/2*1.5.*(kkkd).^(4/3));
%
loglog(kk,slope,kk,VKP,kk(2:end),E(2:end))
ylim([1e-14 10]);
h=legend('Kolmogorov','VKP','Computed');
set(h,'Location','SouthWest')
\end{verbatim}

        \color{lightgray} \begin{verbatim}Warning: Legend not supported for patches with FaceColor 'interp' 
\end{verbatim} \color{black}
    
\includegraphics [width=4in]{spectrum_04.jpg}



\end{document}
    
